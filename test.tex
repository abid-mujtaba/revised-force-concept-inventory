%&preamble

\begin{document}
    \maketitle

    \eline[-2]
    \textbf{Instructions:}
    \begin{enumerate}
        \item Time: 30 min.
        \item Answer ALL questions.
        \item Mark your answers on the provided MCQ Answer Sheet.
    \end{enumerate}
    \begin{center}\rule{0.7\textwidth}{0.5pt}\end{center}

    \question{Q.1}{Two metal balls, H and L, are the same size but H is twice as heavy as L ($m_H = 2 m_L$). The two balls are dropped from \textbf{rest} from the roof of a single story building at the same instant of time. The time it takes the balls to reach the ground below will be:}

         \begin{options}
            \item about half as long for the heavier ball ($\displaystyle t_H = \frac{1}{2} t_L$).
            \item about half as long for the lighter ball ($\displaystyle t_L = \frac{1}{2} t_H$).
            \item about the same for both balls ($t_H = t_L$).
            \item considerably less for the heavier ball ($t_H < t_L$).
            \item considerably less for the lighter ball ($t_L < t_H$).
         \end{options}


    \question{Q.2}{The same two metals balls from Q.1 (H and L) are rolled off of a horizontal table with the same speed. In this situation the horizontal distance ($x$) covered by the balls (from launch to hitting the floor) will be:}

        \begin{options}
            \item about the same for both balls ($x_H = x_L$).
            \item about half as long for the heavier ball ($\displaystyle x_H = \frac{1}{2} x_L$).
            \item about half as long for the lighter ball ($\displaystyle x_L = \frac{1}{2} x_H$).
            \item considerably less for the heavier ball ($x_H < x_L$).
            \item considerably less for the lighter ball ($x_L < x_H$).
        \end{options}


    \question{Q.3}{A stone dropped from the roof a single story building:}

        \begin{options}
            \item reaches a maximum speed soon after release and then falls with constant speed thereafter.
            \item speeds up as it falls because the gravitational attractions gets considerably stronger as the stone gets closer to the earth.
            \item speeds up because of an almost constant force of gravity acting upon it.
            \item falls because of the natural tendency of all objects to rest on the surface of the earth.
            \item falls because of the combined effects of the force of gravity pushing it downward and the force of the air pushing it downward.
        \end{options}


    \question{Q.4}{A large truck collides head-on with a small car. During the collision:}

        \begin{options}
            \item the truck exerts a greater amount of force on the car than the car exerts on the truck.
            \item the car exerts a greater amount of force on the truck than the truck on the car.
            \item neither exerts a force on the other. The car gets smashed simply because it gets in the way of the truck.
            \item the truck exerts a force on the car but the car does not exert a force on the truck.
            \item the truck exerts the same amount of force on the car as the car exerts on the truck.
        \end{options}
\end{document}
