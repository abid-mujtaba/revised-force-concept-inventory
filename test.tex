%&preamble

% Tikz externalization and its external system call cannot be part of the static preamble so it is placed here, after the preamble is loaded.

% By default the externalization runs in batchmode which means no logs are generated on the terminal. Since we are interested in figuring out the error immediately we run the externalized compilation in 'nonstopmode'
% Note: We expliclity declare the fmt (preamble) file in the system call so that the externalized tikz compilation can gain the tremendous speed-up from preamble precompilation
\tikzset{external/system call={pdflatex -fmt=preamble.fmt \tikzexternalcheckshellescape -halt-on-error -interaction=nonstopmode -jobname "\image" "\texsource"}}

\tikzexternalize[prefix=build/]     % Activate image externalization


\begin{document}
    \maketitle

    \eline[-2]
    \textbf{Instructions:}
    \begin{enumerate}
        \item Time: 30 min.
        \item Answer ALL questions.
        \item Mark your answers on the provided MCQ Answer Sheet.
    \end{enumerate}
    \begin{center}\rule{0.7\textwidth}{0.5pt}\end{center}

    \begin{mcq}{Q.1}{Two metal balls, H and L, are the same size but H is twice as heavy as L ($m_H = 2 m_L$). The two balls are dropped from \textbf{rest} from the roof of a single story building at the same instant of time. The time it takes the balls to reach the ground below will be:}

            \item about half as long for the heavier ball ($\displaystyle t_H = \frac{1}{2} t_L$).
            \item about half as long for the lighter ball ($\displaystyle t_L = \frac{1}{2} t_H$).
            \item about the same for both balls ($t_H = t_L$).
            \item considerably less for the heavier ball ($t_H < t_L$).
            \item considerably less for the lighter ball ($t_L < t_H$).
    \end{mcq}
    \answer{1}{C}


    \begin{mcq}{Q.2}{The same two metals balls from Q.1 (H and L) are rolled off of a horizontal table with the same speed. In this situation the horizontal distance ($x$) covered by the balls (from launch to hitting the floor) will be:}

            \item about the same for both balls ($x_H = x_L$).
            \item about half as long for the heavier ball ($\displaystyle x_H = \frac{1}{2} x_L$).
            \item about half as long for the lighter ball ($\displaystyle x_L = \frac{1}{2} x_H$).
            \item considerably less for the heavier ball ($x_H < x_L$).
            \item considerably less for the lighter ball ($x_L < x_H$).
    \end{mcq}
    \answer{2}{A}


    \begin{mcq}{Q.3}{A stone dropped from the roof a single story building:}

            \item reaches a maximum speed soon after release and then falls with constant speed thereafter.
            \item speeds up as it falls because the gravitational attractions gets considerably stronger as the stone gets closer to the earth.
            \item speeds up because of an almost constant force of gravity acting upon it.
            \item falls because of the natural tendency of all objects to rest on the surface of the earth.
            \item falls because of the combined effects of the force of gravity pushing it downward and the force of the air pushing it downward.
    \end{mcq}
    \answer{3}{C}


    \begin{mcq}{Q.4}{A large truck collides head-on with a small car. During the collision:}

            \item the truck exerts a greater amount of force on the car than the car exerts on the truck.
            \item the car exerts a greater amount of force on the truck than the truck on the car.
            \item neither exerts a force on the other. The car gets smashed simply because it gets in the way of the truck.
            \item the truck exerts a force on the car but the car does not exert a force on the truck.
            \item the truck exerts the same amount of force on the car as the car exerts on the truck.
    \end{mcq}
    \answer{4}{E}

    \eline[2]
    The next two question (5 and 6) are based on Fig.~\ref{fig:circular_channel} which shows a horizontal frictionless channel in the shape of a segment (part) of a circle with center at \textbf{O}.
    The channel is placed on top of a horizontal frictionless table and a ball is shot at high speed into end \textbf{P} and exits at end \textbf{R}.
    Air resistance is negligible.

    \begin{figure}[h!]
        \begin{center}
            \eline[]
            \tikzsetnextfilename{circular_channel}

% Note how the size of all node text has been increased
\begin{tikzpicture}[every node/.append style={font=\large}]

    \def\ri{2}
    \def\ro{\ri * 1.3}      % Set outer radius as 30% larger than the inner radius
    \def\ae{-230}           % Angle at end of channel (P)
    \def\aq{{(\ae * 0.5)}}           % Angle for point Q
    \def\rb{0.25}           % Radius of ball

    % Calculate middle of the two radii
    \def\rc{{(\ri + \ro) * 0.5}}        % Note the extra curly braces to encapsulate the arithmetic

    \fill
        (0,0) circle (2pt) node[anchor=south, yshift=3pt] {$O$}
        (\rc, 0) circle (\rb)
        (\ae:\rc) circle (\rb)
        (\aq:\rc) circle (\rb)
    ;

    \draw[semithick]
        (\ri,0) arc (0:\ae:\ri)        % Start arc on x-axis and the swing through to -230 degrees with radius \ri
        (\ro,0) node[anchor=west] {$R$} arc (0:\ae:\ro) node[anchor=south east] {$P$}
        (\aq:{(\rc + 0.7)}) node {$Q$}      % Place text a little further ahead than the ball at Q
    ;

    % Use the arrows.meta tikz library to change the shape and increase size of arrow-head
    % Arcs with arrows to show path of ball inside channel.
    \def\aa{{(\ae + 20)}}
    \def\ab{{(\aq - 20)}}
    \draw[semithick, -{Stealth[scale=1.3]}]
        (\aa:\rc) arc (\aa:\ab:\rc);

    \def\aa{{(\aq + 20)}}
    \def\ab{{(0 - 20)}}
    \draw[semithick, -{Stealth[scale=1.3]}]
        (\aa:\rc) arc (\aa:\ab:\rc)
    ;

    % Arrow pointing out of R
    \draw[semithick, -{Stealth[scale=1.3]}]
        (\rc, 0.4) -- (\rc, 1)
    ;

    % Arrow pointing in to P
    % Create an arrow that points vertically and is placed below the origin
    % Now first the figure is rotated and then shifted to the center of the ball at P placing the arrow correctly
    \draw[semithick, -{Stealth[scale=1.3]}, shift={(-230:\rc)}, rotate=-226]
        (0, -1) -- (0, -0.4)
    ;


\end{tikzpicture}

            \caption{\label{fig:circular_channel} Ball travelling in an horizontal frictionless channel.}
        \end{center}
    \end{figure}

    \begin{mcq}{Q.5}{Consider the following distinct forces:

        \eline[]
        \begin{options}{1.}
            \item A downward force of gravity.
            \item A force exerted by the channel pointing from \textbf{Q} to \textbf{O}.
            \item A force in the direction of the motion.
            \item A force pointing from \textbf{O} to \textbf{Q}.
        \end{options}
        \eline[]

        Which of the forces listed above are acting on the ball when it is within the frictionless channel at the position \textbf{Q}?
    }

        \item 1 only.
        \item 1 and 2.
        \item 1 and 3.
        \item 1, 2, and 3.
        \item 1, 3, and 4.
    \end{mcq}
    \answer{5}{B}

    \newpage    % Added to place question on next page along with its figure
%
    \question{Q.6}{When the ball exits the channel at \textbf{R} which of the paths shown in Fig.~\ref{fig:channel_exit} will it follow as it moves across the horizontal frictionless table?}
    \answer{6}{B}

    \begin{figure}[h!]
        \begin{center}
            \tikzsetnextfilename{channel_exit}

\begin{tikzpicture}[every node/.append style={font=\large}]

    \def\ri{2}
    \def\ro{\ri * 1.3}      % Set outer radius as 30% larger than the inner radius
    \def\ae{-230}           % Angle at end of channel (P)
    \def\aq{{(\ae * 0.5)}}           % Angle for point Q
    \def\rb{0.25}           % Radius of ball

    % Calculate middle of the two radii
    \def\rc{{(\ri + \ro) * 0.5}}        % Note the extra curly braces to encapsulate the arithmetic

    \fill
        (0,0) circle (2pt) node[anchor=south, yshift=3pt] {$O$}
        (\rc, 0) circle (\rb)
    ;

    \draw[semithick]
        (\ri,0) arc (0:\ae:\ri)        % Start arc on x-axis and the swing through to -230 degrees with radius \ri
        (\ro,0) node[anchor=west] {$R$} arc (0:\ae:\ro)
    ;

    % Draw path arrows
    \draw[semithick, -{Stealth[scale=1.3]}]
        (\rc, \rb) coordinate (ps) -- ++ (0, 3) node[anchor=south] {$(B)$};

    \draw[semithick, -{Stealth[scale=1.3]}]
        (ps) arc (0:90:\rc) node[anchor=east] {$(A)$};

    \draw[semithick, -{Stealth[scale=1.3]}]
        (ps) arc (180:120:{(\rc * 1.5)}) node[anchor=west] {$(C)$};

    % Note the use of Bezier curve (using two control points) to create the curve
    % The first control point is straight up so that the curve starts pointing up
    \draw[semithick, -{Stealth[scale=1.3]}]
        (ps) .. controls (\rc, 1) and (3,2) .. ({(\rc * 2)}, 2) node[anchor=west] {$(D)$};

    \draw[semithick, -{Stealth[scale=1.3]}]
        (ps) .. controls (\rc, 1) and (3,1.25) .. ({(\rc * 2)}, 1.25) node[anchor=west] {$(E)$};

\end{tikzpicture}

            \caption{\label{fig:channel_exit} Ball exiting the circular channel.}
        \end{center}
    \end{figure}

    \question{Q.7}{A steel ball is attached to a string and is swung in a horizontal circular path as shown in Fig.~\ref{fig:whirling_ball}. At the point \textbf{P} the string breaks (near the ball). Which path would the ball follow after the string breaks?}
    \answer{7}{B}

    \begin{figure}[h!]
        \begin{center}
            \tikzsetnextfilename{whirling_ball}

\begin{tikzpicture}

    \tikzset{every picture/.style=semithick}        % Set default thickness to 'semithick'
    \tikzset{>={Stealth[scale=1.3]}}      % Set the default style for arrows
    \def\rc{2}      % Radius of circle (path)

    % Use node to include the person-top tikzpicture
    \node[scale=0.5, rotate=-45] at (0, -0.35) {%
        \input{diag/person-top-svg}
    };

    \draw[fill=black] (-45:\rc) coordinate (ball) circle (4pt);
    \draw (0.52,-0.52) -- (ball);

    \draw[dashed, ->] (ball) arc (-45:0:\rc) coordinate (a1);
    \draw[dashed, ->] (a1) arc (0:90:\rc) coordinate (a2);
    \draw[dashed, ->] (a2) arc (90:180:\rc) coordinate (a3);
    \draw[dashed, ->] (a3) arc (180:270:\rc) coordinate (a4);
    \draw[dashed] (a4) arc (270:315:\rc);

    \draw[->] (ball) arc (-45:0:{(\rc * 2)}) node[anchor=south] {$(A)$};
    \draw[->, rotate=45, shift={(ball)}]
        (0,0) -- (3,0) node[anchor=south west] {$(B)$};
    \draw[->] (ball) -- ++ (1.75,0) node[anchor=west, xshift=-3pt] {$(C)$};
    \draw[->] (ball) arc (225:315:\rc) node[anchor=south west, xshift=-4pt, yshift=-4pt] {$(D)$};
    \draw[->, rotate=-45, shift={(ball)}]
        (0,0) -- (3,0) node[anchor=north west] {$(E)$};

\end{tikzpicture}

            \caption{\label{fig:whirling_ball} Path of steel ball (as viewed from above).}
        \end{center}
    \end{figure}


    % \eline[]
    \newpage
    %
    The next four questions (8-11) are based on Fig.~\ref{fig:carrom-kick}.

    The figure shows a carrom striker sliding with constant speed $v_0$ in a straight line from point \textbf{P} to \textbf{Q} on a horizontal frictionless surface (the carrom board).
    Air resistance can be ignored.
    When the striker reaches point \textbf{Q} it is given a quick horizontal kick (jerk) in the direction perpendicular to its motion as depicted by the large arrow.

    If the striker had been stationary at \textbf{Q} then the kick would have caused it to move with speed $v_k$ in the direction of the kick.

    \begin{figure}[h!]
        \begin{center}
            \tikzsetnextfilename{carrom_kick}

\begin{tikzpicture}

    \draw[semithick, dashed, -{Stealth[scale=1.3]}]
        (0,0) node[anchor=east] {$P$} -- (3,0);

    \draw[semithick, dashed]
        (3,0) -- (6,0) node[anchor=west, xshift=10] {$Q$};

    \fill[black]
        (6,0) circle (10pt);

    \fill[black, scale=0.2, shift={(30,-10)}]
        (0,6) -- (2,3) -- (0.8,3) -- (0.8,0) -- (-0.8,0) -- (-0.8,3) -- (-2,3) -- (0,6);

\end{tikzpicture}

        \end{center}
        \caption{\label{fig:carrom-kick} Path of carrom striker just before it is kicked at \textbf{Q}.}
    \end{figure}

    \question{Q.8}{Which of the paths in Fig.~\ref{fig:carrom-paths} would the striker follow after receiving the kick?}

    \begin{figure}[h!]
        \begin{center}
            \tikzsetnextfilename{carrom_paths}

\def\br{0.35}

% Define the common elements of the drawing
\newdrawing{\incoming}{%
    \draw[dashed, ->] (-1.5,0) -- ({(0 - \br)},0);

    \fill[black] (0,0) circle (\br);

    % The latter shift is on the inner scale and pushes the tip to (0,0)
    % The first shift works on the outer scale and shifts the tip down below the circle
    \fill[black, shift={(0,{(-1.5 * \br)})}, scale=0.2, shift={(0,-6)}]
        (0,6) -- (2,3) -- (0.8,3) -- (0.8,0) -- (-0.8,0) -- (-0.8,3) -- (-2,3) -- (0,6);
}


\begin{tikzpicture}

    \tikzset{every picture/.style=semithick}        % Set default thickness to 'semithick'
    \tikzset{>={Stealth[scale=1.3]}}      % Set the default style for arrows

    \newdimen{\ymax}            % Max height of each sub-figure
    \setlength{\ymax}{2cm}

    \newdimen{\sep}             % xshift between successive figures
    \setlength{\sep}{3.75cm}

    \incoming
    \draw[->] (0,0) -- (0,\ymax) node[anchor=south] {$(A)$};

    \begin{scope}[xshift=\sep]
        \incoming
        \draw[->] (0,0) -- (\ymax,\ymax) node[anchor=south] {$(B)$};
    \end{scope}

    \begin{scope}[xshift={(\sep * 2)}]
        \incoming
        % Note: The {\ymax - 0.5} arithmetic requires the 'calc' package to work
        \draw[->] (0,0) -- (0,{\ymax - 0.5cm}) arc (180:90:0.5) -- ++ (0.5,0) node[anchor=south] {$(C)$};
    \end{scope}


    \def\pr{2};         % Radius of curved paths in D and E

    \begin{scope}[xshift={(\sep * 3)}]
        \incoming
        \draw[->] (0,0) arc (-90:-20:\pr) node[anchor=south] {$(D)$};
    \end{scope}

    \begin{scope}[xshift={(\sep * 4)}]
        \incoming
        \draw[->] (0,0) arc (180:110:\pr) node[anchor=south] {$(E)$};
    \end{scope}

\end{tikzpicture}

        \end{center}
        \caption{\label{fig:carrom-paths} Path of the striker after receiving the kick.}
    \end{figure}
    \answer{8}{B}

    \begin{mcq}{Q.9}{The speed of the striker immediate \textbf{after} it is kicked is:}
        \item equal to its speed $v_0$ before it was kicked.
        \item equal to the speed $v_k$ (defined above Q.8).
        \item equal to the sum of the speeds, $(v_0 + v_k)$.
        \item smaller than both $v_0$ and $v_k$.
        \item greater than both $v_0$ and $v_k$ but less than $(v_0 + v_k)$.
    \end{mcq}
    \answer{9}{E}

    \begin{mcq}{Q.10}{Along the frictionless path you have chosen as the answer to Q.8, the speed of the striker after receiving the kick:}
        \item is constant.
        \item continuously increases.
        \item continuously decreases.
        \item increases initially and then decreases.
        \item is constant initially and then decreases.
    \end{mcq}
    \answer{10}{A}

    \begin{mcq}{Q.11}{Consider the following forces:

        \eline[]
        \begin{options}{1.}
            \item A downward force of gravity.
            \item A horizontal force in the direction of motion.
            \item An upward force exerted by the surface (carrom board).
        \end{options}
        \eline[]

        Along the frictionless path you have chosen as the answer to Q.8, the force(s) acting on the striker \textbf{after} receiving the kick are:}
            \item 1 only.
            \item 1 and 2.
            \item 1, 2, and 3.
            \item 1 and 3.
            \item None (No forces act on the puck).
    \end{mcq}
    \answer{11}{D}

    \question{Q.12}{A ball is fired by a cannon from the top off a cliff as shown in Fig.~\ref{fig:cannon}. Which of the paths would the cannon ball most closely follow?}

    \begin{figure}[h!]
        \begin{center}
            \tikzsetnextfilename{cannon}

\begin{tikzpicture}
    \tikzset{every picture/.style=semithick}        % Set default thickness to 'semithick'
    \tikzset{>={Stealth[scale=1.3]}}      % Set the default style for arrows

    \draw (-3,0) -- (0,0) coordinate -- (0,-4) -- (12,-4);

    \node[scale=0.08] at (-1,0.7) {%
        \input{diag/cannon-svg}
    };

    \coordinate (S) at (0,0.7);
    \coordinate (E) at (10,-4);

    \fill[black] (E) circle (4pt);

    \begin{scope}[decoration={markings, mark=at position 0.5 with {\arrow{>}}}]     % All lines in this scope have an arrow in the middle
        \draw[postaction={decorate}] (S) -- node[anchor=north, yshift=-5pt] {$(A)$} (E) coordinate (C);      % Note the postaction that forces draw to use the decoration defined in the scope
    \end{scope}
\end{tikzpicture}

            \caption{\label{fig:cannon} Path of cannon ball fired from atop a cliff.\protect\endnotemark}
        \end{center}
    \end{figure}
    \endnotetext{Cannon icon: https://www.onlinewebfonts.com/icon/7548}
    \answer{12}{B}

    \begin{mcq}{Q.13}{A boy throws a steel ball straight up. From the moment the ball leaves the boy's hand to when the ball hits the ground the forces acting on the ball are:}
        \item a downward force of gravity and a steadily decreasing upward force.
        \item a steadily decreasing upward force while the ball travels up and a steadily increasing downward force of gravity as the ball comes back down and gets closer to the earth.
        \item a constant downward force of gravity along with an upward force that steadily decreases until the ball reaches its highest point; on the way down only a constant downward force of gravity.
        \item a constant downward force of gravity only.
        \item none of the above. The ball falls back to the ground because of its natural tendency to reset on the surface of the earth.
    \end{mcq}
    \answer{13}{D}

    \newpage
    \question{Q.14}{A heavy package is dropped from a supply plane as it flies in a horizontal direction. In Fig.~\ref{fig:plane} choose the path of the falling package as observed by a person on the ground viewing the plane from side on.}

    \begin{figure}[h!]
        \begin{center}
            \tikzsetnextfilename{plane}

\begin{tikzpicture}
   \tikzset{every picture/.style=semithick}        % Set default thickness to 'semithick'
   \tikzset{>={Stealth[scale=1.3]}}      % Set the default style for arrows

   \draw[thick] (-4,-3) -- (4,-3);

   \node[scale=0.2] at (0,0.3) {%
      \input{diag/plane-svg}
   };

   \draw[thick, ->] (1.5,0.2) -- node[anchor=south] {$\vec{v}$} (2.5,0.2);

   \coordinate (P) at (0,0);

   \begin{scope}[decoration={markings, mark=at position 0.5 with {\arrow{>}}}]     % All lines in this scope have an arrow in the middle
      \draw[postaction={decorate}] (P) -- node[anchor=east] {$(B)$} (0,-3);
      \draw[postaction={decorate}] (P) -- node[anchor=east, xshift=-4pt] {$(C)$} (3.5,-3);
   \end{scope}


\end{tikzpicture}

            \caption{\label{fig:plane} Path of package dropped from a flying plane.\protect\endnotemark}
        \end{center}
    \end{figure}
    \endnotetext{Airplane icon made by Freepik from www.flaticon.com}
    \answer{14}{D}

    \eline[2]
    The next two questions (15 and 16) are based on Fig. \ref{fig:vehicles} which shows a broken down truck being pushed by a small car. 

    \begin{figure}[h!]
       \begin{center}
          \tikzsetnextfilename{vehicles}

\begin{tikzpicture}
   \tikzset{every picture/.style=semithick}        % Set default thickness to 'semithick'
   \tikzset{>={Stealth[scale=1.3]}}      % Set the default style for arrows

   \draw (0,0) -- (8,0);

   \node[scale=1] at (5.6,0.77) {\input{diag/truck-svg}};
   \node[scale=0.6] at (2.8,0.42) {\input{diag/car-svg}};

\end{tikzpicture}

          \caption{\label{fig:vehicles} Truck being pushed by a small car.\protect\endnotemark}
       \end{center}
    \end{figure}
    % NOTE: the use of \phantom that creates a box of the size of the included text but hides the text. It is a replacement for \hspace{2em} which doesn't work here.
    \endnotetext{Truck icon: https://commons.wikimedia.org/wiki/File:RWBA\_LKW.svg.\\\phantom{MM}Car icon: https://commons.wikimedia.org/wiki/File:RWBA\_PKW.svg.\\\phantom{MM}Exhaust Gas by Luis Prado from the Noun Project.}

    \eline[-1]
    \begin{mcq}{Q.15}{While the car is pushing the truck and their speed is \textbf{increasing}:}
      \item the amount of force with which the car pushes on the truck is \textbf{equal} to the amount of force with which the truck pushes back on the car.
      \item the amount of force with which the car pushes on the truck is \textbf{smaller} than the amount of force with with which the truck pushes back on the car.
      \item the amount of force with which the car pushes on the truck is \textbf{larger} than the amount of force with which the truck pushes back on the car.
      \item the car's engine is running so the car pushes on the truck, but the truck's engine is not running so the truck \textbf{cannot} push back on the car.
      \item the car and the truck exert \textbf{no} force on each other. The truck is pushed forward because it is in the way of the car.
    \end{mcq}
    \answer{15}{A}

    \begin{mcq}{Q.16}{While the car is pushing the truck and they are moving with a \textbf{constant} speed:}
      \item the amount of force with which the car pushes on the truck is \textbf{equal} to the amount of force with which the truck pushes back on the car.
      \item the amount of force with which the car pushes on the truck is \textbf{smaller} than the amount of force with with which the truck pushes back on the car.
      \item the amount of force with which the car pushes on the truck is \textbf{larger} than the amount of force with which the truck pushes back on the car.
      \item the car's engine is running so the car pushes on the truck, but the truck's engine is not running so the truck \textbf{cannot} push back on the car.
      \item the car and the truck exert \textbf{no} force on each other. The truck is pushed forward because it is in the way of the car.
    \end{mcq}
    \answer{16}{A}

    \begin{mcq}{Q.17}{An elevator is being lifted up using a steel cable as shown in Fig. \ref{fig:elevator}. All frictional effects can be ignored. If the elevator is rising at constant speed then:}
        \item the upward force exerted by the cable is \textbf{greater} than the downward force of gravity.
        \item the upward force exerted by the cable is \textbf{equal} to the downward force of gravity.
        \item the upward force exerted by the cable is \textbf{smaller} than the downward force of gravity.
        \item only an upward force exerted by the cable is present.
        \item none of the above. The elevator goes up because the cable is shortened, not because of an upward force exerted by the cable. 
    \end{mcq}
    \answer{17}{B}

    \begin{figure}[h!]
       \begin{center}
          \tikzsetnextfilename{elevator}

\begin{tikzpicture}
    \tikzset{every picture/.style=semithick}        % Set default thickness to 'semithick'
    \tikzset{>={Stealth[scale=1.3]}}      % Set the default style for arrows

    % Elevator shaft and top housing
    \draw
        (-2.5,0) -- (-1,0) -- (-1,-4.5)
        (2.5,0) -- (1,0) -- (1,-4.5)
        (2,0) |- (0.1,1) coordinate (HR)
        (-2,0) |- (-0.1,1) coordinate (HL)
    ;

    % Cable
    \draw[thick] 
        (0,1.25) coordinate (T) -- (0,-1.25) 
    ;

    % Elevator box
    \draw[very thick]
        (-0.9, -1.25) rectangle (0.9, -3.5)
    ;

    % Up arrow in elevator
    \draw[ultra thick, ->] (0, -3) -- (0,-1.75);

    % Motor core
    \fill[black] ($ (T) + (0.15,0) $) coordinate (C) circle[radius=0.15];

    \begin{scope}
        \clip (-1,1) rectangle (1,2);       % Only anything in this scope that lies within the clipped rectangle will be visible
        \draw (C) circle[radius=0.35];
    \end{scope}

\end{tikzpicture}

          \caption{\label{fig:elevator} Steel cable lifting elevator at constant speed.}
       \end{center}
    \end{figure}

    \begin{mcq}{Q.18}{Fig. \ref{fig:swing} shows a girl on a swing who happens to be at a point $A$ which is \textbf{lower} than the high-point of her swing. Consider the following distinct forces acting on the girl:

        \eline[]
        \begin{options}{1.}
            \item A downward force of gravity.
            \item A force exerted by the rope pointing from $A$ to $O$.
            \item A force in the direction of the girl's motion.
            \item A force pointing from $O$ to $A$.
        \end{options}
        \eline[]

        Which of the forces (listed above) are acting on the girl when she is at position $A$?
    } 
        \item 1 only.
        \item 1 and 2.
        \item 1 and 3.
        \item 1, 2, and 3.
        \item 1, 3, and 4.
    \end{mcq}
    \answer{18}{B}

    \begin{figure}[h!]
        \begin{center}
            % TODO: Add figure.
            \tikzsetnextfilename{swing}

\begin{tikzpicture}
    \tikzset{every picture/.style=semithick}        % Set default thickness to 'semithick'
    \tikzset{>={Stealth[scale=1.3]}}      % Set the default style for arrows
    
    \fill[black] (0,0) circle[radius=0.1] node[anchor=south, yshift=2pt] {$O$};

    \coordinate (P) at (-120:4);

    \node[scale=0.1, xshift=30pt, yshift=30pt] at (P) {%
        \input{diag/person-swing-svg}
    };

    \node at (-120:4.5) {$A$};
    
    \draw[thick] (0,0) -- (P);

    \draw[thick, ->, rotate=-30] (P) ++ (0.5,0) -- ++ (1, 0);
    \draw[dashed] (-124:4.1) arc (-124:-140:4.1);

\end{tikzpicture}

            \caption{\label{fig:swing} Girl on a swing.}
        \end{center}
    \end{figure}

    \begin{mcq}{Q.25}{A woman exerts a constant \textbf{horizontal} force on a large box. As a result, the box moves across a horizontal force with constant speed.
   
        \eline[]
        The force applied by the woman:
    }
        \item has the \textbf{same} magnitude as the weight of the box.
        \item is \textbf{greater} than the weight of the box.
        \item has the \textbf{same} magnitude as the force which resists the motion of the box.
        \item is \textbf{greater} than the force which resists the motion of the box.
        \item is \textbf{greater} than either weight of the box or the force which resists its motion.
    \end{mcq}
    \answer{25}{C}

    \begin{mcq}{Q.26}{If the woman in the previous question (Q.25) doubles the constant horizontal force that she exerts on the box, then the box moves:}
        \item with a constant speed that is \textbf{double} the speed in the previous question.
        \item with a constant speed that is \textbf{greater} than the speed in the previous question.
        \item for a while with \textbf{constant} speed, then with \textbf{increasing} speed afterwards.
        \item for a while with \textbf{increasing} speed, then with \textbf{constant} speed afterwards.
        \item with a continuously \textbf{increasing} speed.
    \end{mcq}

    \vfill      % Pushes the endnotes to the end of the page
    \theendnotes

\end{document}
