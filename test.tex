%&preamble

% Tikz externalization and its external system call cannot be part of the static preamble so it is placed here, after the preamble is loaded.

% By default the externalization runs in batchmode which means no logs are generated on the terminal. Since we are interested in figuring out the error immediately we run the externalized compilation in 'nonstopmode'
% Note: We expliclity declare the fmt (preamble) file in the system call so that the externalized tikz compilation can gain the tremendous speed-up from preamble precompilation
\tikzset{external/system call={pdflatex -fmt=preamble.fmt \tikzexternalcheckshellescape -halt-on-error -interaction=nonstopmode -jobname "\image" "\texsource"}}

\tikzexternalize[prefix=build/]     % Activate image externalization


\begin{document}
    \maketitle

    \eline[-2]
    \textbf{Instructions:}
    \begin{enumerate}
        \item Time: 30 min.
        \item Answer ALL questions.
        \item Mark your answers on the provided MCQ Answer Sheet.
    \end{enumerate}
    \begin{center}\rule{0.7\textwidth}{0.5pt}\end{center}

    \begin{mcq}{Q.1}{Two metal balls, H and L, are the same size but H is twice as heavy as L ($m_H = 2 m_L$). The two balls are dropped from \textbf{rest} from the roof of a single story building at the same instant of time. The time it takes the balls to reach the ground below will be:}

            \item about half as long for the heavier ball ($\displaystyle t_H = \frac{1}{2} t_L$).
            \item about half as long for the lighter ball ($\displaystyle t_L = \frac{1}{2} t_H$).
            \item about the same for both balls ($t_H = t_L$).
            \item considerably less for the heavier ball ($t_H < t_L$).
            \item considerably less for the lighter ball ($t_L < t_H$).
    \end{mcq}
    \answer{1}{C}


    \begin{mcq}{Q.2}{The same two metals balls from Q.1 (H and L) are rolled off of a horizontal table with the same speed. In this situation the horizontal distance ($x$) covered by the balls (from launch to hitting the floor) will be:}

            \item about the same for both balls ($x_H = x_L$).
            \item about half as long for the heavier ball ($\displaystyle x_H = \frac{1}{2} x_L$).
            \item about half as long for the lighter ball ($\displaystyle x_L = \frac{1}{2} x_H$).
            \item considerably less for the heavier ball ($x_H < x_L$).
            \item considerably less for the lighter ball ($x_L < x_H$).
    \end{mcq}
    \answer{2}{A}


    \begin{mcq}{Q.3}{A stone dropped from the roof a single story building:}

            \item reaches a maximum speed soon after release and then falls with constant speed thereafter.
            \item speeds up as it falls because the gravitational attractions gets considerably stronger as the stone gets closer to the earth.
            \item speeds up because of an almost constant force of gravity acting upon it.
            \item falls because of the natural tendency of all objects to rest on the surface of the earth.
            \item falls because of the combined effects of the force of gravity pushing it downward and the force of the air pushing it downward.
    \end{mcq}
    \answer{3}{C}


    \begin{mcq}{Q.4}{A large truck collides head-on with a small car. During the collision:}

            \item the truck exerts a greater amount of force on the car than the car exerts on the truck.
            \item the car exerts a greater amount of force on the truck than the truck on the car.
            \item neither exerts a force on the other. The car gets smashed simply because it gets in the way of the truck.
            \item the truck exerts a force on the car but the car does not exert a force on the truck.
            \item the truck exerts the same amount of force on the car as the car exerts on the truck.
    \end{mcq}
    \answer{4}{E}

    \eline[2]
    The next two question (5 and 6) are based on Fig.~\ref{fig:circular_channel} which shows a horizontal frictionless channel in the shape of a segment (part) of a circle with center at \textbf{O}.
    The channel is placed on top of a horizontal frictionless table and a ball is shot at high speed into end \textbf{P} and exits at end \textbf{R}.
    Air resistance is negligible.

    \begin{figure}[h!]
        \begin{center}
            \eline[]
            \tikzsetnextfilename{circular_channel}

% Note how the size of all node text has been increased
\begin{tikzpicture}[every node/.append style={font=\large}]

    \def\ri{2}
    \def\ro{\ri * 1.3}      % Set outer radius as 30% larger than the inner radius
    \def\ae{-230}           % Angle at end of channel (P)
    \def\aq{{(\ae * 0.5)}}           % Angle for point Q
    \def\rb{0.25}           % Radius of ball

    % Calculate middle of the two radii
    \def\rc{{(\ri + \ro) * 0.5}}        % Note the extra curly braces to encapsulate the arithmetic

    \fill
        (0,0) circle (2pt) node[anchor=south, yshift=3pt] {$O$}
        (\rc, 0) circle (\rb)
        (\ae:\rc) circle (\rb)
        (\aq:\rc) circle (\rb)
    ;

    \draw[semithick]
        (\ri,0) arc (0:\ae:\ri)        % Start arc on x-axis and the swing through to -230 degrees with radius \ri
        (\ro,0) node[anchor=west] {$R$} arc (0:\ae:\ro) node[anchor=south east] {$P$}
        (\aq:{(\rc + 0.7)}) node {$Q$}      % Place text a little further ahead than the ball at Q
    ;

    % Use the arrows.meta tikz library to change the shape and increase size of arrow-head
    % Arcs with arrows to show path of ball inside channel.
    \def\aa{{(\ae + 20)}}
    \def\ab{{(\aq - 20)}}
    \draw[semithick, -{Stealth[scale=1.3]}]
        (\aa:\rc) arc (\aa:\ab:\rc);

    \def\aa{{(\aq + 20)}}
    \def\ab{{(0 - 20)}}
    \draw[semithick, -{Stealth[scale=1.3]}]
        (\aa:\rc) arc (\aa:\ab:\rc)
    ;

    % Arrow pointing out of R
    \draw[semithick, -{Stealth[scale=1.3]}]
        (\rc, 0.4) -- (\rc, 1)
    ;

    % Arrow pointing in to P
    % Create an arrow that points vertically and is placed below the origin
    % Now first the figure is rotated and then shifted to the center of the ball at P placing the arrow correctly
    \draw[semithick, -{Stealth[scale=1.3]}, shift={(-230:\rc)}, rotate=-226]
        (0, -1) -- (0, -0.4)
    ;


\end{tikzpicture}

            \caption{\label{fig:circular_channel} Ball travelling in an horizontal frictionless channel.}
        \end{center}
    \end{figure}

    \begin{mcq}{Q.5}{Consider the following distinct forces:

        \eline[]
        \begin{options}{1.}
            \item A downward force of gravity.
            \item A force exerted by the channel pointing from \textbf{Q} to \textbf{O}.
            \item A force in the direction of the motion.
            \item A force pointing from \textbf{O} to \textbf{Q}.
        \end{options}
        \eline[]

        Which of the forces listed above are acting on the ball when it is within the frictionless channel at the position \textbf{Q}?
    }

        \item 1 only.
        \item 1 and 2.
        \item 1 and 3.
        \item 1, 2, and 3.
        \item 1, 3, and 4.
    \end{mcq}
    \answer{5}{B}

    \question{Q.6}{When the ball exits the channel at \textbf{R} which of the paths shown in Fig.~\ref{fig:channel_exit} will it follow as it moves across the horizontal frictionless table?}

    \begin{figure}[h!]
        \begin{center}
            \tikzsetnextfilename{channel_exit}

\begin{tikzpicture}[every node/.append style={font=\large}]

    \def\ri{2}
    \def\ro{\ri * 1.3}      % Set outer radius as 30% larger than the inner radius
    \def\ae{-230}           % Angle at end of channel (P)
    \def\aq{{(\ae * 0.5)}}           % Angle for point Q
    \def\rb{0.25}           % Radius of ball

    % Calculate middle of the two radii
    \def\rc{{(\ri + \ro) * 0.5}}        % Note the extra curly braces to encapsulate the arithmetic

    \fill
        (0,0) circle (2pt) node[anchor=south, yshift=3pt] {$O$}
        (\rc, 0) circle (\rb)
    ;

    \draw[semithick]
        (\ri,0) arc (0:\ae:\ri)        % Start arc on x-axis and the swing through to -230 degrees with radius \ri
        (\ro,0) node[anchor=west] {$R$} arc (0:\ae:\ro)
    ;

    % Draw path arrows
    \draw[semithick, -{Stealth[scale=1.3]}]
        (\rc, \rb) coordinate (ps) -- ++ (0, 3) node[anchor=south] {$(B)$};

    \draw[semithick, -{Stealth[scale=1.3]}]
        (ps) arc (0:90:\rc) node[anchor=east] {$(A)$};

    \draw[semithick, -{Stealth[scale=1.3]}]
        (ps) arc (180:120:{(\rc * 1.5)}) node[anchor=west] {$(C)$};

    % Note the use of Bezier curve (using two control points) to create the curve
    % The first control point is straight up so that the curve starts pointing up
    \draw[semithick, -{Stealth[scale=1.3]}]
        (ps) .. controls (\rc, 1) and (3,2) .. ({(\rc * 2)}, 2) node[anchor=west] {$(D)$};

    \draw[semithick, -{Stealth[scale=1.3]}]
        (ps) .. controls (\rc, 1) and (3,1.25) .. ({(\rc * 2)}, 1.25) node[anchor=west] {$(E)$};

\end{tikzpicture}

            \caption{\label{fig:channel_exit} Ball exiting the circular channel.}
        \end{center}
    \end{figure}

    \question{Q.7}{A steel ball is attached to a string and is swung in a horizontal circular path as shown in Fig.~\ref{fig:whirling_ball}. At the point \textbf{P} the string breaks (near the ball). Which path would the ball follow after the string breaks?}

    \begin{figure}[h!]
        \begin{center}
            \tikzsetnextfilename{whirling_ball}

\begin{tikzpicture}

    \tikzset{every picture/.style=semithick}        % Set default thickness to 'semithick'
    \tikzset{>={Stealth[scale=1.3]}}      % Set the default style for arrows
    \def\rc{2}      % Radius of circle (path)

    % Use node to include the person-top tikzpicture
    \node[scale=0.5, rotate=-45] at (0, -0.35) {%
        \input{diag/person-top-svg}
    };

    \draw[fill=black] (-45:\rc) coordinate (ball) circle (4pt);
    \draw (0.52,-0.52) -- (ball);

    \draw[dashed, ->] (ball) arc (-45:0:\rc) coordinate (a1);
    \draw[dashed, ->] (a1) arc (0:90:\rc) coordinate (a2);
    \draw[dashed, ->] (a2) arc (90:180:\rc) coordinate (a3);
    \draw[dashed, ->] (a3) arc (180:270:\rc) coordinate (a4);
    \draw[dashed] (a4) arc (270:315:\rc);

    \draw[->] (ball) arc (-45:0:{(\rc * 2)}) node[anchor=south] {$(A)$};
    \draw[->, rotate=45, shift={(ball)}]
        (0,0) -- (3,0) node[anchor=south west] {$(B)$};
    \draw[->] (ball) -- ++ (1.75,0) node[anchor=west, xshift=-3pt] {$(C)$};
    \draw[->] (ball) arc (225:315:\rc) node[anchor=south west, xshift=-4pt, yshift=-4pt] {$(D)$};
    \draw[->, rotate=-45, shift={(ball)}]
        (0,0) -- (3,0) node[anchor=north west] {$(E)$};

\end{tikzpicture}

            \caption{\label{fig:whirling_ball} Path of steel ball (as viewed from above).}
        \end{center}
    \end{figure}

\end{document}
