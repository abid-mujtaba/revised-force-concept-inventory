\tikzsetnextfilename{channel_exit}

\begin{tikzpicture}[every node/.append style={font=\large}]

    \def\ri{2}
    \def\ro{\ri * 1.3}      % Set outer radius as 30% larger than the inner radius
    \def\ae{-230}           % Angle at end of channel (P)
    \def\aq{{(\ae * 0.5)}}           % Angle for point Q
    \def\rb{0.25}           % Radius of ball

    % Calculate middle of the two radii
    \def\rc{{(\ri + \ro) * 0.5}}        % Note the extra curly braces to encapsulate the arithmetic

    \fill
        (0,0) circle (2pt) node[anchor=south, yshift=3pt] {$O$}
        (\rc, 0) circle (\rb)
    ;

    \draw[semithick]
        (\ri,0) arc (0:\ae:\ri)        % Start arc on x-axis and the swing through to -230 degrees with radius \ri
        (\ro,0) node[anchor=west] {$R$} arc (0:\ae:\ro)
    ;

    % Draw path arrows
    \draw[semithick, -{Stealth[scale=1.3]}]
        (\rc, \rb) coordinate (ps) -- ++ (0, 3) node[anchor=south] {$(B)$};

    \draw[semithick, -{Stealth[scale=1.3]}]
        (ps) arc (0:90:\rc) node[anchor=east] {$(A)$};

    \draw[semithick, -{Stealth[scale=1.3]}]
        (ps) arc (180:120:{(\rc * 1.5)}) node[anchor=west] {$(C)$};

    % Note the use of Bezier curve (using two control points) to create the curve
    % The first control point is straight up so that the curve starts pointing up
    \draw[semithick, -{Stealth[scale=1.3]}]
        (ps) .. controls (\rc, 1) and (3,2) .. ({(\rc * 2)}, 2) node[anchor=west] {$(D)$};

    \draw[semithick, -{Stealth[scale=1.3]}]
        (ps) .. controls (\rc, 1) and (3,1.25) .. ({(\rc * 2)}, 1.25) node[anchor=west] {$(E)$};

\end{tikzpicture}
