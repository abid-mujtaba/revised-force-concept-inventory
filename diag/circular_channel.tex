\tikzsetnextfilename{circular_channel}

% Note how the size of all node text has been increased
\begin{tikzpicture}[every node/.append style={font=\large}]

    \def\ri{2}
    \def\ro{\ri * 1.3}      % Set outer radius as 30% larger than the inner radius
    \def\ae{-230}           % Angle at end of channel (P)
    \def\aq{{(\ae * 0.5)}}           % Angle for point Q
    \def\rb{0.25}           % Radius of ball

    % Calculate middle of the two radii
    \def\rc{{(\ri + \ro) * 0.5}}        % Note the extra curly braces to encapsulate the arithmetic

    \fill
        (0,0) circle (2pt) node[anchor=south, yshift=3pt] {$O$}
        (\rc, 0) circle (\rb)
        (\ae:\rc) circle (\rb)
        (\aq:\rc) circle (\rb)
    ;

    \draw[semithick]
        (\ri,0) arc (0:\ae:\ri)        % Start arc on x-axis and the swing through to -230 degrees with radius \ri
        (\ro,0) node[anchor=west] {$R$} arc (0:\ae:\ro) node[anchor=south east] {$P$}
        (\aq:{(\rc + 0.7)}) node {$Q$}      % Place text a little further ahead than the ball at Q
    ;

    % Use the arrows.meta tikz library to change the shape and increase size of arrow-head
    % Arcs with arrows to show path of ball inside channel.
    \def\aa{{(\ae + 20)}}
    \def\ab{{(\aq - 20)}}
    \draw[semithick, -{Stealth[scale=1.3]}]
        (\aa:\rc) arc (\aa:\ab:\rc);

    \def\aa{{(\aq + 20)}}
    \def\ab{{(0 - 20)}}
    \draw[semithick, -{Stealth[scale=1.3]}]
        (\aa:\rc) arc (\aa:\ab:\rc)
    ;

    % Arrow pointing out of R
    \draw[semithick, -{Stealth[scale=1.3]}]
        (\rc, 0.4) -- (\rc, 1)
    ;

    % Arrow pointing in to P
    % Create an arrow that points vertically and is placed below the origin
    % Now first the figure is rotated and then shifted to the center of the ball at P placing the arrow correctly
    \draw[semithick, -{Stealth[scale=1.3]}, shift={(-230:\rc)}, rotate=-226]
        (0, -1) -- (0, -0.4)
    ;


\end{tikzpicture}
